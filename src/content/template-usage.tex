% !TeX spellcheck = hu_HU
% !TeX encoding = UTF-8
% !TeX program = xelatex
%----------------------------------------------------------------------------
\chapter{A \LaTeX-sablon használata}
%----------------------------------------------------------------------------

Ebben a fejezetben röviden, implicit módon bemutatjuk a sablon használatának módját, ami azt jelenti, hogy sablon használata ennek a dokumentumnak a forráskódját tanulmányozva válik teljesen világossá. Amennyiben a szoftver-keretrendszer telepítve van, a sablon alkalmazása és a dolgozat szerkesztése \LaTeX-ben a sablon segítségével tapasztalataink szerint jóval hatékonyabb, mint egy WYSWYG (\emph{What You See is What You Get}) típusú szövegszerkesztő esetén (pl. Microsoft Word, OpenOffice).

%----------------------------------------------------------------------------
\section{Címkék és hivatkozások}
%----------------------------------------------------------------------------
A \LaTeX~dokumentumban címkéket (\verb+\label+) rendelhetünk ábrákhoz, táblázatokhoz, fejezetekhez, listákhoz, képletekhez stb. Ezekre a dokumentum bármely részében hivatkozhatunk, a hivatkozások automatikusan feloldásra kerülnek.

A sablonban makrókat definiáltunk a hivatkozások megkönnyítéséhez. Ennek megfelelően minden ábra (\emph{figure}) címkéje \verb+fig:+ kulcsszóval kezdődik, míg minden táblázat (\emph{table}), képlet (\emph{equation}), fejezet (\emph{section}) és lista (\emph{listing}) rendre a \verb+tab:+, \verb+eq:+, \verb+sec:+ és \verb+lst:+ kulcsszóval kezdődik, és a kulcsszavak után tetszőlegesen választott címke használható. Ha ezt a konvenciót betartjuk, akkor az előbbi objektumok számára rendre a \verb+\figref+, \verb+\tabref+, \verb+\eqref+, \verb+\sectref+ és \verb+\listref+ makrókkal hivatkozhatunk. A makrók paramétere a címke, amelyre hivatkozunk (a kulcsszó nélkül). Az összes említett hivatkozástípus, beleértve az \verb+\url+ kulcsszóval bevezetett web-hivatkozásokat is a  \verb+hyperref+\footnote{Segítségével a dokumentumban megjelenő hivatkozások nem csak dinamikussá válnak, de színezhetők is, bővebbet erről a csomag dokumentációjában találunk. Ez egyúttal egy példa lábjegyzet írására.} csomagnak köszönhetően aktívak a legtöbb PDF-nézegetőben, rájuk kattintva a dokumentum megfelelő oldalára ugrik a PDF-néző vagy a megfelelő linket megnyitja az alapértelmezett böngészővel. A \verb+hyperref+ csomag a kimeneti PDF-dokumentumba könyvjelzőket is készít a tartalomjegyzékből. Ez egy szintén aktív tartalomjegyzék, amelynek elemeire kattintva a nézegető behozza a kiválasztott fejezetet.

%----------------------------------------------------------------------------
\section{Ábrák és táblázatok}
%----------------------------------------------------------------------------
Használjunk vektorgrafikus ábrákat, ha van rá módunk. PDFLaTeX használata esetén PDF formátumú ábrákat lehet beilleszteni könnyen, az EPS (PostScript) vektorgrafikus képformátum beillesztését a PDFLaTeX közvetlenül nem támogatja (de lehet konvertálni, lásd később). Ha vektorgrafikus formában nem áll rendelkezésünkre az ábra, akkor a  veszteségmentes PNG, valamint a veszteséges JPEG formátumban érdemes elmenteni.  Figyeljünk arra, hogy ilyenkor a képek felbontása elég nagy legyen ahhoz, hogy nyomtatásban is megfelelő minőséget nyújtson (legalább 300 dpi javasolt). A dokumentumban felhasznált képfájlokat a dokumentum forrása mellett érdemes tartani, archiválni, mivel ezek hiányában a dokumentum nem fordul újra. Ha lehet, a vektorgrafikus képeket vektorgrafikus formátumban is érdemes elmenteni az újrafelhasználhatóság (az átszerkeszthetőség) érdekében.

Kapcsolási rajzok legtöbbször kimásolhatók egy vektorgrafikus programba (pl. CorelDraw) és onnan nagyobb felbontással raszterizálva kimenthatők PNG formátumban. Ugyanakkor kiváló ábrák készíthetők Microsoft Visio vagy hasonló program használatával is: Visio-ból az ábrák közvetlenül PDF-be is menthetők.

Lehetőségeink Matlab ábrák esetén:
\begin{itemize}
	\item Képernyőlopás (\emph{screenshot}) is elfogadható minőségű lehet a dokumentumban, de általában jobb felbontást is el lehet érni más módszerrel.
	\item A Matlab ábrát a \verb+File/Save As+ opcióval lementhetjük PNG formátumban (ugyanaz itt is érvényes, mint korábban, ezért nem javasoljuk).
	\item A Matlab ábrát az \verb+Edit/Copy figure+ opcióval kimásolhatjuk egy vektorgrafikus programba is és onnan nagyobb felbontással raszterizálva kimenthatjük PNG formátumban (nem javasolt).
	\item Javasolt megoldás: az ábrát a \verb+File/Save As+ opcióval EPS \emph{vektorgrafikus} formátumban elmentjük, PDF-be konvertálva beillesztjük a dolgozatba.
\end{itemize}
Az EPS kép az \verb+epstopdf+ programmal\footnote{a korábban említett \LaTeX-disztribúciókban megtalálható} konvertálható PDF formátumba. Célszerű egy batch-fájlt készíteni az összes EPS ábra lefordítására az alábbi módon (ez Windows alatt működik).
\begin{lstlisting}
@echo off
for %%j in (*.eps) do (
echo converting file "%%j"
epstopdf "%%j"
)
echo done .
\end{lstlisting}

Egy ilyen parancsfájlt (\verb+convert.cmd+) elhelyeztük a sablon \verb+figures\eps+ könyvtárába, így a felhasználónak csak annyi a dolga, hogy a \verb+figures\eps+ könyvtárba kimenti az EPS formátumú vektorgrafikus képet, majd lefuttatja a \verb+convert.cmd+ parancsfájlt, ami PDF-be konvertálja az EPS fájlt.

Ezek után a PDF-ábrát ugyanúgy lehet a dokumentumba beilleszteni, mint a PNG-t vagy a JPEG-et. A megoldás előnye, hogy a lefordított dokumentumban is vektorgrafikusan tárolódik az ábra, így a mérete jóval kisebb, mintha raszterizáltuk volna beillesztés előtt. Ez a módszer minden -- az EPS formátumot ismerő -- vektorgrafikus program (pl. CorelDraw) esetén is használható.

A képek beillesztésére \az+\refstruc{sec:LatexTools}ben mutattunk be példát (\refstruc{fig:TeXstudio}). Az előző mondatban egyúttal az automatikusan feloldódó ábrahivatkozásra is láthatunk példát. Több képfájlt is beilleszthetünk egyetlen ábrába. Az egyes képek közötti horizontális és vertikális margót metrikusan szabályozhatjuk (\refstruc{fig:HVSpaces}). Az ábrák elhelyezését számtalan tipográfiai szabály egyidejű teljesítésével a fordító maga végzi, a dokumentum írója csak preferenciáit jelezheti a fordító felé (olykor ez bosszúságot is okozhat, ilyenkor pl. a kép méretével lehet játszani).

\begin{figure}[!ht]
	\centering
	\includegraphics[width=67mm, keepaspectratio]{figures/TeXstudio.png}\hspace{1cm}
	\includegraphics[width=67mm, keepaspectratio]{figures/TeXstudio.png}\\\vspace{5mm}
	\includegraphics[width=67mm, keepaspectratio]{figures/TeXstudio.png}\hspace{1cm}
	\includegraphics[width=67mm, keepaspectratio]{figures/TeXstudio.png}
	\caption{Több képfájl beillesztése esetén térközöket is érdemes használni.}
	\label{fig:HVSpaces}
\end{figure}

A táblázatok használatára \aref{tab:TabularExample}~táblázat mutat példát. A táblázatok formázásához hasznos tanácsokat találunk a \verb+booktabs+ csomag dokumentációjában.

\begin{table}[ht]
	\footnotesize
	\centering
	\begin{tabular}{ l c c }
		\toprule
		Órajel & Frekvencia & Cél pin \\
		\midrule
		CLKA & 100 MHz & FPGA CLK0\\
		CLKB & 48 MHz  & FPGA CLK1\\
		CLKC & 20 MHz  & Processzor\\
		CLKD & 25 MHz  & Ethernet chip \\
		CLKE & 72 MHz  & FPGA CLK2\\
		XBUF & 20 MHz  & FPGA CLK3\\
		\bottomrule
	\end{tabular}
	\caption{Az órajel-generátor chip órajel-kimenetei.}
	\label{tab:TabularExample}
\end{table}


%----------------------------------------------------------------------------
\section{Felsorolások és listák}
%----------------------------------------------------------------------------
Számozatlan felsorolásra mutat példát a jelenlegi bekezdés:
\begin{itemize}
	\item \emph{első bajusz:} ide lehetne írni az első elem kifejését,
	\item \emph{második bajusz:} ide lehetne írni a második elem kifejését,
	\item \emph{ez meg egy szakáll:} ide lehetne írni a harmadik elem kifejését.
\end{itemize}

Számozott felsorolást is készíthetünk az alábbi módon:
\begin{enumerate}
	\item \emph{első bajusz:} ide lehetne írni az első elem kifejését, és ez a kifejtés így néz ki, ha több sorosra sikeredik,
	\item \emph{második bajusz:} ide lehetne írni a második elem kifejését,
	\item \emph{ez meg egy szakáll:} ide lehetne írni a harmadik elem kifejését.
\end{enumerate}
A felsorolásokban sorok végén vessző, az utolsó sor végén pedig pont a szokásos írásjel. Ez alól kivételt képezhet, ha az egyes elemek több teljes mondatot tartalmaznak.

Listákban a dolgozat szövegétől elkülönítendő kódrészleteket, programsorokat, pszeudo-kódokat jeleníthetünk meg (\ref{lst:Example}.~kódrészlet).
\begin{lstlisting}[caption=A fenti számozott felsorolás \LaTeX-forráskódja,label=lst:Example]
\begin{enumerate}
	\item \emph{els(*@ő@*) bajusz:} ide lehetne írni az els(*@ő@*) elem kifejését,
	és ez a kifejtés így néz ki, ha több sorosra sikeredik,
	\item \emph{második bajusz:} ide lehetne írni a második elem kifejését,
	\item \emph{ez meg egy szakáll:} ide lehetne írni a harmadik elem kifejését.
\end{enumerate}
\end{lstlisting}
A lista keretét, háttérszínét, egész stílusát megválaszthatjuk. Ráadásul különféle programnyelveket és a nyelveken belül kulcsszavakat is definiálhatunk, ha szükséges. Erről bővebbet a \verb+listings+ csomag hivatalos leírásában találhatunk.

%----------------------------------------------------------------------------
\section{Képletek}
%----------------------------------------------------------------------------
Ha egy formula nem túlságosan hosszú, és nem akarjuk hivatkozni a szövegből, mint például a $e^{i\pi}+1=0$ képlet, \emph{szövegközi képletként} szokás leírni. Csak, hogy másik példát is lássunk, az $U_i=-d\Phi/dt$ Faraday-törvény a $\rot E=-\frac{dB}{dt}$ differenciális alakban adott Maxwell-egyenlet felületre vett integráljából vezethető le. Látható, hogy a \LaTeX-fordító a sorközöket betartja, így a szöveg szedése esztétikus marad szövegközi képletek használata esetén is.

Képletek esetén az általános konvenció, hogy a kisbetűk skalárt, a kis félkövér betűk ($\mathbf{v}$) oszlopvektort -- és ennek megfelelően $\mathbf{v}^T$ sorvektort -- a kapitális félkövér betűk ($\mathbf{V}$) mátrixot jelölnek. Ha ettől el szeretnénk térni, akkor az alkalmazni kívánt jelölésmódot célszerű külön alfejezetben definiálni. Ennek megfelelően, amennyiben $\mathbf{y}$ jelöli a mérések vektorát, $\mathbf{\vartheta}$ a paraméterek vektorát és $\hat{\mathbf{y}}=\mathbf{X}\vartheta$ a paraméterekben lineáris modellt, akkor a \emph{Least-Squares} értelemben optimális paraméterbecslő $\hat{\mathbf{\vartheta}}_{LS}=(\mathbf{X}^T\mathbf{X})^{-1}\mathbf{X}^T\mathbf{y}$ lesz.

Emellett kiemelt, sorszámozott képleteket is megadhatunk, ennél az \verb+equation+ és a \verb+eqnarray+ környezetek helyett a korszerűbb \verb+align+ környezet alkalmazását javasoljuk (több okból, különféle problémák elkerülése végett, amelyekre most nem térünk ki). Tehát
\begin{align}
\dot{\mathbf{x}}&=\mathbf{A}\mathbf{x}+\mathbf{B}\mathbf{u},\\
\mathbf{y}&=\mathbf{C}\mathbf{x},
\end{align}
ahol $\mathbf{x}$ az állapotvektor, $\mathbf{y}$ a mérések vektora és $\mathbf{A}$, $\mathbf{B}$ és $\mathbf{C}$ a rendszert leíró paramétermátrixok. Figyeljük meg, hogy a két egyenletben az egyenlőségjelek egymáshoz igazítva jelennek meg, mivel a mindkettőt az \& karakter előzi meg a kódban. Lehetőség van számozatlan kiemelt képlet használatára is, például
\begin{align}
\dot{\mathbf{x}}&=\mathbf{A}\mathbf{x}+\mathbf{B}\mathbf{u},\nonumber\\
\mathbf{y}&=\mathbf{C}\mathbf{x}\nonumber.
\end{align}
Mátrixok felírására az $\mathbf{A}\mathbf{x}=\mathbf{b}$ inhomogén lineáris egyenlet részletes kifejtésével mutatunk példát:
\begin{align}
\begin{bmatrix}
a_{11} & a_{12} & \dots & a_{1n}\\
a_{21} & a_{22} & \dots & a_{2n}\\
\vdots & \vdots & \ddots & \vdots\\
a_{m1} & a_{m2} & \dots & a_{mn}
\end{bmatrix}
\begin{pmatrix}x_1\\x_2\\\vdots\\x_n\end{pmatrix}=
\begin{pmatrix}b_1\\b_2\\\vdots\\b_m\end{pmatrix}.
\end{align}
A \verb+\frac+ utasítás hatékonyságát egy általános másodfokú tag átviteli függvényén keresztül mutatjuk be, azaz
\begin{align}
W(s)=\frac{A}{1+2T\xi s+s^2T^2}.
\end{align}
A matematikai mód minden szimbólumának és képességének a bemutatására természetesen itt nincs lehetőség, de gyors referenciaként hatékonyan használhatók a következő linkek:\\
\indent\url{http://www.artofproblemsolving.com/LaTeX/AoPS_L_GuideSym.php},\\
\indent\url{http://www.ctan.org/tex-archive/info/symbols/comprehensive/symbols-a4.pdf},\\
\indent\url{ftp://ftp.ams.org/pub/tex/doc/amsmath/short-math-guide.pdf}.\\
Ez pedig itt egy magyarázat, hogy miért érdemes \verb+align+ környezetet használni:\\
\indent\url{http://texblog.net/latex-archive/maths/eqnarray-align-environment/}.

%----------------------------------------------------------------------------
\section{Irodalmi hivatkozások}
\label{sec:HowtoReference}
%----------------------------------------------------------------------------
Egy \LaTeX~dokumentumban az irodalmi hivatkozások definíciójának két módja van. Az egyik a \verb+\thebibliograhy+ környezet használata a dokumentum végén, az \verb+\end{document}+ lezárás előtt.
\begin{lstlisting}
\begin{thebibliography}{9}

\bibitem{Lamport94} Leslie Lamport, \emph{\LaTeX: A Document Preparation System}.
Addison Wesley, Massachusetts, 2nd Edition, 1994.

\end{thebibliography}
\end{lstlisting}

Ezek után a dokumentumban a \verb+\cite{Lamport94}+ utasítással hivatkozhatunk a forrásra. A fenti megadás viszonylag kötetlen, a szerző maga formázza az irodalomjegyzéket (ami gyakran inkonzisztens eredményhez vezet).

Egy sokkal professzionálisabb módszer a BiB\TeX{} használata, ezért ez a sablon is ezt támogatja. Ebben az esetben egy külön szöveges adatbázisban definiáljuk a forrásmunkákat, és egy külön stílusfájl határozza meg az irodalomjegyzék kinézetét. Ez, összhangban azzal, hogy külön formátumkonvenció határozza meg a folyóirat-, a könyv-, a konferenciacikk- stb. hivatkozások kinézetét az irodalomjegyzékben (a sablon használata esetén ezzel nem is kell foglalkoznia a hallgatónak, de az eredményt célszerű ellenőrizni). felhasznált hivatkozások adatbázisa egy \verb+.bib+ kiterjesztésű szöveges fájl, amelynek szerkezetét a \Aref{lst:Bibtex} kódrészlet demonstrálja. A forrásmunkák bevitelekor a sor végi vesszők külön figyelmet igényelnek, mert hiányuk a BiB\TeX-fordító hibaüzenetét eredményezi. A forrásmunkákat típus szerinti kulcsszó vezeti be (\verb+@book+ könyv, \verb+@inproceedings+ konferenciakiadványban megjelent cikk, \verb+@article+ folyóiratban megjelent cikk, \verb+@techreport+ valamelyik egyetem gondozásában megjelent műszaki tanulmány, \verb+@manual+ műszaki dokumentáció esetén stb.). Nemcsak a megjelenés stílusa, de a kötelezően megadandó mezők is típusról-típusra változnak. Egy jól használható referencia a \url{http://en.wikipedia.org/wiki/BibTeX} oldalon található.

\begin{lstlisting}[caption=Példa szöveges irodalomjegyzék-adatbázisra Bib\TeX{} használata esetén.,label=lst:Bibtex]
@book{Wettl04,
  author    = {Ferenc Wettl and Gyula Mayer and Péter Szabó},
  publisher = {Panem Könyvkiadó},
  title     = {\LaTeX~kézikönyv},
  year      = {2004},
}

@article{Candy86,
  author       = {James C. Candy},
  journaltitle = {{IEEE} Trans.\ on Communications},
  month        = {01},
  note         = {\doi{10.1109/TCOM.1986.1096432}},
  number       = {1},
  pages        = {72--76},
  title        = {Decimation for Sigma Delta Modulation},
  volume       = {34},
  year         = {1986},
}

@inproceedings{Lee87,
  author    = {Wai L. Lee and Charles G. Sodini},
  booktitle = {Proc.\ of the IEEE International Symposium on Circuits and Systems},
  location  = {Philadelphia, PA, USA},
  month     = {05~4--7},
  pages     = {459--462},
  title     = {A Topology for Higher Order Interpolative Coders},
  vol       = {2},
  year      = {1987},
}

@thesis{KissPhD,
  author      = {Peter Kiss},
  institution = {Technical University of Timi\c{s}oara, Romania},
  month       = {04},
  title       = {Adaptive Digital Compensation of Analog Circuit Imperfections for Cascaded Delta-Sigma Analog-to-Digital Converters},
  type        = {phdthesis},
  year        = {2000},
}

@manual{Schreier00,
  author       = {Richard Schreier},
  month        = {01},
  note         = {\url{http://www.mathworks.com/matlabcentral/fileexchange/}},
  organization = {Oregon State University},
  title        = {The Delta-Sigma Toolbox v5.2},
  year         = {2000},
}

@misc{DipPortal,
  author       = {{Budapesti Műszaki és Gazdaságtudományi Egyetem Villamosmérnöki és Informatikai Kar}},
  howpublished = {\url{http://diplomaterv.vik.bme.hu/}},
  title        = {Diplomaterv portál (2011. február 26.)},
}

@incollection{Mkrtychev:1997,
  author    = {Mkrtychev, Alexey},
  booktitle = {Logical Foundations of Computer Science},
  doi       = {10.1007/3-540-63045-7_27},
  editor    = {Adian, Sergei and Nerode, Anil},
  isbn      = {978-3-540-63045-6},
  pages     = {266-275},
  publisher = {Springer Berlin Heidelberg},
  series    = {Lecture Notes in Computer Science},
  title     = {Models for the logic of proofs},
  url       = {http://dx.doi.org/10.1007/3-540-63045-7_27},
  volume    = {1234},
  year      = {1997},
}
\end{lstlisting}

A stílusfájl egy \verb+.sty+ kiterjesztésű fájl, de ezzel lényegében nem kell foglalkozni, mert vannak beépített stílusok, amelyek jól használhatók. Ez a sablon a BiB\TeX-et használja, a hozzá tartozó adatbázisfájl a \verb+mybib.bib+ fájl. Megfigyelhető, hogy az irodalomjegyzéket a dokumentum végére (a \verb+\end{document}+ utasítás elé) beillesztett \verb+\bibliography{mybib}+ utasítással hozhatjuk létre, a stílusát pedig ugyanitt a  \verb+\bibliographystyle{plain}+ utasítással adhatjuk meg. Ebben az esetben a \verb+plain+ előre definiált stílust használjuk (a sablonban is ezt állítottuk be). A \verb+plain+ stíluson kívül természetesen számtalan más előre definiált stílus is létezik. Mivel a \verb+.bib+ adatbázisban ezeket megadtuk, a BiB\TeX-fordító is meg tudja különböztetni a szerzőt a címtől és a kiadótól, és ez alapján automatikusan generálódik az irodalomjegyzék a stílusfájl által meghatározott stílusban.

Az egyes forrásmunkákra a szövegből továbbra is a \verb+\cite+ paranccsal tudunk hivatkozni, így \aref{lst:Bibtex}.~kódrészlet esetén a hivatkozások rendre \verb+\cite{Wettl04}+, \verb+\cite{Candy86}+, \verb+\cite{Lee87}+, \verb+\cite{KissPhD}+, \verb+\cite{Schreirer00}+,
\verb+\cite{Mkrtychev:1997}+ és \verb+\cite{DipPortal}+. Az egyes forrásmunkák sorszáma az irodalomjegyzék bővítésekor változhat. Amennyiben az aktuális számhoz illeszkedő névelőt szeretnénk használni, használjuk az \verb+\acite{}+ parancsot.

Az irodalomjegyzékben alapértelmezésben csak azok a forrásmunkák jelennek meg, amelyekre található hivatkozás a szövegben, és ez így alapvetően helyes is, hiszen olyan forrásmunkákat nem illik az irodalomjegyzékbe írni, amelyekre nincs hivatkozás.

Mivel a fordítási folyamat során több lépésben oldódnak fel a szimbólumok, ezért gyakran többször is le kell fordítani a dokumentumot. Ilyenkor ez első 1-2 fordítás esetleg szimbólum-feloldásra vonatkozó figyelmeztető üzenettel zárul. Ha hibaüzenettel zárul bármelyik fordítás, akkor nincs értelme megismételni, hanem a hibát kell megkeresni. A \verb+.bib+ fájl megváltoztatáskor sokszor nincs hatása a változtatásnak azonnal, mivel nem mindig fut újra a BibTeX fordító. Ezért célszerű a változtatás után azt manuálisan is lefuttatni (TeXstudio esetén \verb+Tools/Bibliography+).

Hogy a szövegbe ágyazott hivatkozások kinézetét demonstráljuk, itt most sorban meghivatkozzuk a \cite{Wettl04}, \cite{Candy86}, \cite{Lee87}, \cite{KissPhD}, \cite{Schreier00} és \acite{Mkrtychev:1997}\footnote{Informatikai témában gyakran hivatkozunk cikkeket a Springer LNCS valamely kötetéből, ez a hivatkozás erre mutat egy helyes példát.} forrásmunkát, valamint \acite{DipPortal} weboldalt.

Megjegyzendő, hogy az ékezetes magyar betűket is tartalmazó \verb+.bib+ fájl az \verb+inputenc+ csomaggal betöltött \verb+latin2+ betűkészlet miatt fordítható. Ugyanez a \verb+.bib+ fájl hibaüzenettel fordul egy olyan dokumentumban, ami nem tartalmazza a \verb+\usepackage[latin2]{inputenc}+ sort. Speciális igény esetén az irodalmi adatbázis általánosabb érvényűvé tehető, ha az ékezetes betűket speciális latex karakterekkel helyettesítjük a \verb+.bib+ fájlban, pl. á helyett \verb+\'{a}+-t vagy ő helyett \verb+\H{o}+-t írunk.

Irodalomhivatkozásokat célszerű először olyan szolgáltatásokban keresni, ahol jó minőségű bejegyzések találhatók (pl. ACM Digital Library,\footnote{\url{https://dl.acm.org/}} DBLP,\footnote{\url{http://dblp.uni-trier.de/}} IEEE Xplore,\footnote{\url{http://ieeexplore.ieee.org/}} SpringerLink\footnote{\url{https://link.springer.com/}}) és csak ezek után használni kevésbé válogatott forrásokat (pl. Google Scholar\footnote{\url{http://scholar.google.com/}}). A jó minőségű bejegyzéseket is érdemes megfelelően tisztítani.\footnote{\url{https://github.com/FTSRG/cheat-sheets/wiki/BibTeX-Fixing-entries-from-common-sources}} A sablon angol nyelvű változatában használt \texttt{plainnat} beállítás egyik sajátossága, hogy a cikkhez generált hivatkozás a cikk DOI-ját és URL-jét is tartalmazza, ami gyakran duplikátumhoz vezet -- érdemes tehát a DOI-kat tartalmazó URL mezőket törölni. 

%----------------------------------------------------------------------------
\section{A dolgozat szerkezete és a forrásfájlok}
%----------------------------------------------------------------------------
A diplomatervsablonban a TeX fájlok két alkönyvtárban helyezkednek el. Az \verb+include+ könyvtárban azok szerepelnek, amiket tipikusan nem kell szerkesztenünk, ezek a sablon részei (pl. címoldal). A \verb+content+ alkönyvtárban pedig a saját munkánkat helyezhetjük el. Itt érdemes az egyes fejezeteket külön \TeX{} állományokba rakni.

A diplomatervsablon (a kari irányelvek szerint) az alábbi fő fejezetekből áll:
\begin{enumerate}
	\item 1 oldalas \emph{tájékoztató} a szakdolgozat/diplomaterv szerkezetéről (\verb+include/guideline.tex+), ami a végső dolgozatból törlendő,
	\item \emph{feladatkiírás} (\verb+include/project.tex+), a dolgozat nyomtatott verzójában ennek a helyére kerül a tanszék által kiadott, a tanszékvezető által aláírt feladatkiírás, a dolgozat elektronikus verziójába pedig a feladatkiírás egyáltalán ne kerüljön bele, azt külön tölti fel a tanszék a diplomaterv-honlapra,
	\item \emph{címoldal} (\verb+include/titlepage.tex+),
	\item \emph{tartalomjegyzék} (\verb+thesis.tex+),
	\item a diplomatervező \emph{nyilatkozat}a az önálló munkáról (\verb+include/declaration.tex+),
	\item 1-2 oldalas tartalmi \emph{összefoglaló} magyarul és angolul, illetve elkészíthető még további nyelveken is (\verb+content/abstract.tex+),
	\item \emph{bevezetés}: a feladat értelmezése, a tervezés célja, a feladat indokoltsága, a diplomaterv felépítésének rövid összefoglalása (\verb+content/introduction.tex+),
	\item sorszámmal ellátott \emph{fejezetek}: a feladatkiírás pontosítása és részletes elemzése, előzmények (irodalomkutatás, hasonló alkotások), az ezekből levonható következtetések, a tervezés részletes leírása, a döntési lehetőségek értékelése és a választott megoldások indoklása, a megtervezett műszaki alkotás értékelése, kritikai elemzése, továbbfejlesztési lehetőségek,
	\item esetleges \emph{köszönetnyilvánítás}ok (\verb+content/acknowledgement.tex+),
	\item részletes és pontos \emph{irodalomjegyzék} (ez a sablon esetében automatikusan generálódik a \verb+thesis.tex+ fájlban elhelyezett \verb+\bibliography+ utasítás hatására, \az+\refstruc{sec:HowtoReference}ban leírtak szerint),
	\item \emph{függelékek} (\verb+content/appendices.tex+).
\end{enumerate}

A sablonban a fejezetek a \verb+thesis.tex+ fájlba vannak beillesztve \verb+\include+ utasítások segítségével. Lehetőség van arra, hogy csak az éppen szerkesztés alatt álló \verb+.tex+ fájlt fordítsuk le, ezzel lerövidítve a fordítási folyamatot. Ezt a lehetőséget az alábbi kódrészlet biztosítja a \verb+thesis.tex+ fájlban.
\begin{lstlisting}
\includeonly{
	guideline,%
	project,%
	titlepage,%
	declaration,%
	abstract,%
	introduction,%
	chapter1,%
	chapter2,%
	chapter3,%
	acknowledgement,%
	appendices,%
}
\end{lstlisting}

Ha az alábbi kódrészletben az egyes sorokat a \verb+%+ szimbólummal kikommentezzük, akkor a megfelelő \verb+.tex+ fájl nem fordul le. Az oldalszámok és a tartalomjegyék természetesen csak akkor billennek helyre, ha a teljes dokumentumot lefordítjuk.

%----------------------------------------------------------------------------
\newpage
\section{Alapadatok megadása}
%----------------------------------------------------------------------------
A diplomaterv alapadatait (cím, szerző, konzulens, konzulens titulusa) a \verb+thesis.tex+ fájlban lehet megadni.

%----------------------------------------------------------------------------
\section{Új fejezet írása}
%----------------------------------------------------------------------------
A főfejezetek külön \verb+content+ könyvtárban foglalnak helyet. A sablonhoz 3 fejezet készült. További főfejezeteket úgy hozhatunk létre, ha új \TeX~fájlt készítünk a fejezet számára, és a \verb+thesis.tex+ fájlban, a \verb+\include+ és \verb+\includeonly+ utasítások argumentumába felvesszük az új \verb+.tex+ fájl nevét.


%----------------------------------------------------------------------------
\section{Definíciók, tételek, példák}
%----------------------------------------------------------------------------

\begin{definition}[Fluxuskondenzátor térerőssége]
Lorem ipsum dolor sit amet, consectetur adipiscing elit, sed do eiusmod tempor incididunt ut labore et dolore magna aliqua. Ut enim ad minim veniam, quis nostrud exercitation ullamco laboris nisi ut aliquip ex ea commodo consequat.
\end{definition}

\begin{example}
Példa egy példára. Duis aute irure dolor in reprehenderit in voluptate velit esse cillum dolore eu fugiat nulla pariatur. Excepteur sint occaecat cupidatat non proident, sunt in culpa qui officia deserunt mollit anim id est laborum.
\end{example}

\begin{theorem}[Kovács tétele]
Duis aute irure dolor in reprehenderit in voluptate velit esse cillum dolore eu fugiat nulla pariatur. Excepteur sint occaecat cupidatat non proident, sunt in culpa qui officia deserunt mollit anim id est laborum.
\end{theorem}
