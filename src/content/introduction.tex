%----------------------------------------------------------------------------
\chapter{\bevezetes}
%----------------------------------------------------------------------------

A bevezető tartalmazza a diplomaterv-kiírás elemzését, 
történelmi előzményeit, a feladat indokoltságát (a motiváció leírását), 
az eddigi megoldásokat, 
és ennek tükrében a hallgató megoldásának összefoglalását.

A bevezető szokás szerint a diplomaterv felépítésével záródik, azaz annak rövid leírásával, hogy melyik fejezet mivel foglalkozik. \cite{Mkrtychev:1997}

\section{Motiváció}
Mai világunk elképzelhetetlen lenne hatékony és gyors közlekedés nélkül.
Ez főleg érvényes a tömegközlekedési eszközökre, legyen az repülőgép, vonat, metró, villamos vagy más egyéb tömegközlekedési eszköz.
Nélkülük szinte képtelenség lenne a nagyvárosi lét.

Napjainkban szinte az élet minden terén egyre komplexebb és komplexebb rendszereket fejlesztenek ki.
Így van ez a közlekedési szektor minden részén, benne a vasútiparban is.
Emellett a megnövekedett bonyolultság mellett kell lehető leggyorsabban kifejleszteni az adott rendszereket ügyelve a növekedő biztonsági elvárásokra.

A biztonság, rendelkezésre állás és költséghatékonyság problémája jellemzi leginkább a mai globális vasútipart és a vasút környezetét.
Ezért a kereslet azokért a rendszerek, amelyek képesek magasfokú biztonságot, rendelkezésreállást és költséghatékonyságot tudnak nyújtani növekedni fog.
Különösen igaz ez a vasúti fékrendszerekre.

A növekedő komplexitás és biztonsági követelmények miatt egyre inkább nehézkessé válik a hagyományos dokumentum-alapú fejlesztési módszertan.
Ezért dolgozatomban szeretnék bemutatni a vasúti fékrendszeren keresztül egy modell-alapú rendszerfejlesztési megközelítést, bemutatva annak pozitív hatását a komplexitás kezelése érdekében.
Továbbá a dolgozat második felében a modellezett rendszer egy vezérlő egységének biztonsági vizsgálatával folytatom.

\section{A dolgozat felépítése}
A dolgozat \ref{chap:hatter}. fejezetében ismertetem a modell-alapú rendszerfejlesztés methodikáját és kis betekintést nyújtok a SysML nyelvbe.
A \ref{chap:irodalom}. fejezetben irodalmi kutatást végzek a biztonsági analízis módszertanához.
A \ref{chap:cs}. fejezet írja le az esettanulmány hátterét, míg a \ref{chap:model}. fejezet az esettanulmány során elkészített vasúti fékrendszer modelljét tartalmazza.
A \ref{chap:analysis}. fejezet tartalmazza az esettanulmány során végrehajtott biztonsági vizsgálatot.