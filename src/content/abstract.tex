\pagenumbering{roman}
\setcounter{page}{1}

\selecthungarian

%----------------------------------------------------------------------------
% Abstract in Hungarian
%----------------------------------------------------------------------------
\chapter*{Kivonat}\addcontentsline{toc}{chapter}{Kivonat}
Modern világunkban elengedhetetlen a bitonságos és megbízható utazás.
Egyre öszetettebb közlekedési rendszereknél létfontosságú lépést tartani a növekedett komplexitás melletti gyors fejlesztés és főleg a fejlesztés biztonságosságának bizonyítása.

Ebben a dolgozatban szeretném bemutatni a modell-alapú rendszertervezés megközelítését, annak előnyeit egy esettanulmány keretében készített valóságszerű vasúti férendszeren keresztül.
Az elkészített rendszerterv alapján szeretnék rávilágítani a funkcionális biztonság fontosságára.
A diplomamunka célja feltárni az irodalomban fellelhető biztonsági analízis módszereit. 
Ezek bemutatása, majd az alkalmasnak választott módszerek részletes bemutatása az esettanulmány során tervezett rendszer alapján.


\vfill
%\selectenglish


%----------------------------------------------------------------------------
% Abstract in English
%----------------------------------------------------------------------------
%\chapter*{Abstract}\addcontentsline{toc}{chapter}{Abstract}

%This document is a \LaTeX-based skeleton for BSc/MSc~theses of students at the Electrical Engineering and Informatics Faculty, Budapest University of Technology and Economics. The usage of this skeleton is optional. It has been tested with the \emph{TeXLive} \TeX~implementation, and it requires the PDF-\LaTeX~compiler.


%\vfill
\cleardoublepage

\selectthesislanguage

\newcounter{romanPage}
\setcounter{romanPage}{\value{page}}
\stepcounter{romanPage}