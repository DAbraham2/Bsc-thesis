\pagenumbering{roman}
\setcounter{page}{1}

\selecthungarian

%----------------------------------------------------------------------------
% Abstract in Hungarian
%----------------------------------------------------------------------------
\chapter*{Kivonat}\addcontentsline{toc}{chapter}{Kivonat}
Modern világunkban elengedhetetlen a biztonságos és megbízható utazás.
Egyre öszetettebb közlekedési rendszereknél létfontosságú lépést tartani a növekedett komplexitás melletti gyors fejlesztés és főleg a fejlesztés biztonságosságának bizonyítása.
Manapság elképzelhetetlen lenne szabványok nélkül kritikus rendszert fejleszteni.
Nincs ez másképp a vasútiparban sem.

A vasúti fékrendszer az egyik legkritikusabb eleme a biztonságos közlegedésnek, főleg a városi közlekedésben aktívan résztvevő villamosoknál.
Ebben a dolgozatban szeretném bemutatni a modell-alapú rendszertervezés megközelítését, annak előnyeit egy esettanulmány keretében készített valóságszerű vasúti férendszeren keresztül.
Az elkészített rendszerterv alapján szeretnék rávilágítani a funkcionális biztonság fontosságára.
A diplomamunka célja feltárni az irodalomban fellelhető biztonsági analízis módszereit.
Feltárni Top-down és Bottom-up módszereket egyaránt az analízis lehető legszélesebb palettájának elvégzése érdekében.
Ezek bemutatása, majd az alkalmasnak választott módszerek részletes bemutatása az esettanulmány során tervezett rendszer alapján.



\vfill
\selectenglish


%----------------------------------------------------------------------------
% Abstract in English
%----------------------------------------------------------------------------
\chapter*{Abstract}\addcontentsline{toc}{chapter}{Abstract}

Safe and reliable travel is essential in out modern world.
In Transportation systems, getting more and more complex it is vital that one can keep up development time and especially safety with the humongous complexity.
Safety-critical development is unimaginable without standards nowadays.
Railway systems are no different either.

Railways brake systems are one of the most critical parts of a train, predominantly in tramways that are an active participant of city commute.
In this study, I would like to show some of the benefits of model-based systems engineering in a case study about an imaginary railways brake system based on reality.
I would like to highlight the importance of function safety with the designed system.
Goal of this study is to explore various methodologies for functional safety.
Explore a wide palette of methods, Top-down and Bottom-up as well to make a comprehensive range of safety analysis.
Showcase their possibilities, and then select a variety of methods to discover their usecases on the designed system during the case study.


\vfill
\cleardoublepage

\selectthesislanguage

\newcounter{romanPage}
\setcounter{romanPage}{\value{page}}
\stepcounter{romanPage}