\chapter{Irodalmi kutatás}\label{chap:irodalom}
\section{FIT Allokáció} \label{sec:fit_allocation}
Az ipari rendszerek egyre összetettebbekké váltak az évek során. 
Emellett mostanság egyre több ilyen rendszer tartalmaz elektronikát és szoftvert, 
tehát a funkcionális biztonságnak folyamatosan nő a fontossága. (Schabe, 2018 \cite{Schabe})

A safety integrity level (SIL) egy diszkrét érték ami meghatározza a használandó módszereket, technikákat a véletlenszerű és szisztematikus hibák elkerülése érdekében.
A SIL-ek koncepciója már több szabvány rendszerben ki lett fejlesztve.
Ezek között legismertebb szabványok az IEC 61508, DEF-STAN-0056, EN 50126, EN 50128, EN 50129 és még sok más.

A SIL-nek két fő aspektusa van:
\begin{enumerate}
    \item Egy cél hibaráta, amit a rendszernek nem szabad meghaladni, hogy tudja kezelni a véletlen hibákat
    \item Módszerek és technikák halmaza, amit a szisztematikus hibákat kezeli 
\end{enumerate}

Itt fontos megjegyezni, hogy szoftverben csak és kizárólag szisztematikus hibákat vesznek figyelembe és nincs megadva cél hibaráta. Ez abból adódik, hogy a szoftverben nincs véletlenszerű hiba.

\subsection{Különböző SIL-ek}
A \ref{tab:SILs} táblázat példát ad a SIL-ek és a tűrhető hibaráta kapcsolatára, ahogy három szabvány, az IEC 61508, az EN 50129 és a DEF-STAN-0056 definiálja.

A tűrhető hibaráta (THR) egy eszköz veszélyes hibáinak maximális rátája, amit a szabvány definiál bizonyos safety integrity szinthez.
Látni kell azt, hogy bár az IEC és EN szabványoknál azonosak a THR értékek, a DEF-STAN-0056-ban eltér.
Ezért a szabványok közti átjárás nem mindig triviális.
Attól még, hogy a THR értékek hasonlóak az IEC és EN szabványok között, az rendszer szintű hibaelkerülő módszerek különböznek, ezért ezek a SIL-ek sem ugyan azok.
\begin{table}[ht]
	\footnotesize
	\centering
	\begin{tabular}{ l c c }
		\toprule
		SIL & IEC 61508/EN 50129 & DEF-STAN-0056 \\
		\midrule
		1 & \(10^{-6}/h \leq\) THR \(< 10^{-5}/h\) & Frequent \(\approx 10^{-2}/h\)\\
		2 & \(10^{-7}/h \leq\) THR \(< 10^{-6}/h\)  & Probable \(\approx 10^{-4}/h\)\\
		3 & \(10^{-8}/h \leq\) THR \(< 10^{-7}/h\)  & Occasional \(\approx 10^{-6}/h\)\\
		4 & \(10^{-9}/h \leq\) THR \(< 10^{-8}/h\)  & Remote \(\approx 10^{-8}/h\) \\
		\bottomrule
	\end{tabular}
	\caption{SIL értékek több szabvány és THR szerint}
	\label{tab:SILs}
\end{table}

\subsection{Safety integrity szintek kombinációja}
Ebben a részben a különböző szabványok SIL szintjeinek kombinációját mutatom be.

\subsubsection{DEF-STAN-0056}
A szabvány a következő szabályokat definiálja a 7.4.4. rész 5.8. táblázatában:
\begin{itemize}
	\item Két SIL3 eszköz párhuzamos kombinációjaként létrejövő rendszer SIL4-es besorolású lesz.
	\item Két SIL2 eszköz párhuzamos kombinációjaként létrejövő rendszer SIL3-es besorolású lesz.
	\item Két SIL1 eszköz párhuzamos kombinációjaként létrejövő rendszer SIL2-es besorolású lesz.
	\item Két eszköz párhuzamos kombinációjaként - ahol az eszközök rendre SIL X és SIL Y besorolásúak - létrejövő rendszer SIL értéke SIL max(x,y).
\end{itemize}

Az olvasót a szabvány figyelmezteti, hogy ne keverje össze ezeket a szabályokat az EN 20129\cite{EN50129} által definiált safety integrity szintekkel.

Továbbá jelen esetben a "párhuzamos kombináció" azt jelenti, hogy a két eszköz vagy funkció úgy van társítva, hogy a veszélyes hiba kiváltásához mind a két eszköz hibája szükséges.

\subsubsection{IEC 61508}
A szabvány nem rendelkezik előre definiált szabályokkal, mint a fenti esetben, de ad némi lehetőséget magasabb integritási szint elérésére kombinációk által.
Az általános szabály a következő (lásd IEC 61508-2, 7.4.4.2.4. rész):
\begin{center}
	Selecting the channel with the highest safety integrity level that has been achieved for the safety funciton under consideration and the adding N safety integrity levels to determing the maximum safety integrity level for the overall combination of the subsystem.
\end{center}

Itt N a párhuzamos kombinációban résztvevő elemek hardverhiba tűrése, azaz hány veszélyes hibát tud a rendszer tolerálni.
Továbbá a rendszerek/elemek között is létezik megkülönböztetés, A és B típus. 

Egy elem A típusúnak mondható, ha a biztonsági funkció eléréséhez a következők teljesülnek:
\begin{enumerate}
	\item A komponens alkotórészeinek összes hibamódja jól körülhatárolt.
	\item Hiba esetén a komponens viselkedése teljes mértékben meghatározható.
	\item Létezik elegendő megbízható meghibásodási adat, ami bizonyítja az állított hibarátát.
\end{enumerate}
Minden más elem/rendszer a B kategóriába kerül.

A követelményeknek való megfelelésből is látszik, hogy a szabvány nem ad egyszerű szabályokat a integritási szintek elosztásáról. 
Nem csak a rendszerek elrendezése és ez által a hardveres hibatűrése határozza meg a SIL-t, hanem még a rendszer biztonsági hiba hányada (Safe Failure Fraction [SSF]) is.
Az SSF meghatározásának módját a szabvány C függelékében definiálja.

\subsubsection{SIRF 400}
A Sicherheitsrichtlinie Farhrzeug (SIRF) a németországi rendelet a vasúti járművek biztonságára. 
Németországban könnyű hivatkozni erre a dokumentumra, de az ország határain kívül nem biztosított az autómatikus megfelelése.

A dokumentum SIL allokáció problémájára a következő elveket adja.

Két alrandszer soros összeköttetése (például: hibafában VAGY kapuval összekötve) esetén, a legkisebb SIL érték határozza meg az összekapcsolt rendszer integritási szintjét.

A párhuzamos kombinációkhoz a következő szabályok adottak:
\begin{enumerate}
	\item Egy SIL > 0 rendszer nem állítható össze SIL 0 elemekből.
	\item Egy integritási szint elengedhető maximum egy szinttel egy ÉS kapu alatt.
	\item Kizárás a 2. pont alól: Egy ág teljesen átveszi a biztonsági funkciót.
	\item Kizárás a 2. pont alól: Common cause failure analízis kivitelezésre került.
	\item A 4. pont esetében egy megfelelő szisztematikus módszert (FMEA, HAZOP, etc.) a hibafa legalsóbb szintjéig kell alkalmazni, hogy bebizonyosodjon a CCF kizárásának lehetősége.
\end{enumerate}
Az \ref{fig:sirfSIL}. ábrán látható a SIRF által megengedett kombinációk. 
Az ábrákon zöld szín jelzi a megengedett, prios szín a tiltott kombinációkat és a fehér jelöli azokat, amikhez további elemzést kell végrehajtani, ami megmutatja az alegységek függetlenségét.

\begin{figure}[!ht]
	\centering
	\includegraphics[width=67mm, keepaspectratio]{figures/sirf_sil1.png}\hspace{1cm}
	\includegraphics[width=67mm, keepaspectratio]{figures/sirf_sil2.png}\\\vspace{1cm}
	\includegraphics[width=67mm, keepaspectratio]{figures/sirf_sil3.png}\hspace{1cm}
	\includegraphics[width=67mm, keepaspectratio]{figures/sirf_sil4.png}
	\caption{Megengedett kombinációk az integritás szinteknek megfelelően \emph{Forrás}: SIRF 400}
	\label{fig:sirfSIL}
\end{figure}

\subsubsection{EN 50129}
Hazard to last independent function.
Ezután TFFR és SIL Allocation.
A funkciók alrészeinél a TFFR további bontása, de megmarad a SIL, performed by a suitable representation of the combination logic (RBD, FTA, Binary Decision D, Markov, Petri, etc).

SIL allocation, functions controlling the same hazard, SILs shall be allocated according to the TFFR.
For a function controlling multiple hazards, if derivation of TFFR apportionment and SIL allocation are preformed separately for each hazard, most restrictive shall apply.
\section{Top-down módszerek}
\subsection{Funkcionális analízis}
A funkcionális analízis (FA) egy alapvető módszer a rendszer kritikus funkciói megértéséhez és tervezéséhez.
Az FA elvégzése nélkülözhetetlen a RAMS menedzsmentben és a rendszertervezésben.
A vizsgálat célja, hogy információt adjon arról, ami befolyásolja a rendszer funkcióit és alapot nyújtson a RAMS menedzsmenthez.

Az FA még a specifikáció, modellezés, szimuláció, validáció és verifikáció lépéseinél is alkalmazható.
Ezért a funkcionális analízis gyakran fontos módszerként alkalmazzák a rendszer funkcionális struktúrájának meghatározására.
A funkcionális analízist általában az alábbi két megközelítésben alkalmazzák:
\begin{itemize}
    \item Struktúrált Analízis és Design módszer (SADT\footnote{Structured Analysis and Design Technique})
    \item Funkcionális Analízis Rendszer módszer (FAST\footnote{Functional Analysis System Technique})
\end{itemize}
SADT-ot több iparágban is alkalmazzák. 
Ez egy diagramszerű koncepció, mely segítségével megérthető és leírhatók a rendszer funkcionális viselkedési és interfészei.
A módszer számos elemet szolgáltat az aktivitások és adatfolyamok reprezentálására és nyilakat ezek kapcsolataihoz, ahhogy az alábbi \ref{fig:sadt} ábrán látható, ami egy példa az SADT-ra.

\begin{figure}
    \footnotesize
    \centering
    \includegraphics[width=100mm, keepaspectratio]{figures/roh2007.jpg}
    \caption{SADT modell (Forrás: Roh, 2007\cite{doi:10.1080/13675560701478240})}
    \label{fig:sadt}
\end{figure}

A FAST Charles Bytheway fejlesztette ki 1964-ben. Ezt a módszert is sok iparág használja.
A FAST-ot azokban a szituációkban lehetséges felhasználni, ahol a funkciókat logikai sorozatban lehet ábrázoloni, ezáltal priorizálni őket és megvizsgálni a függőségeit.
Ez a módszer nem alkalmas funkcionális problémák megoldására, inkább feltárni a rendszer alapvető funkcionális karakterisztikáit.
Az \ref{fig:fast} ábra egy példa a FAST modellre Kaufmann (1982) által.

\begin{figure}
    \footnotesize
    \centering
    \includegraphics[width=150mm, keepaspectratio]{figures/fast.png}
    \caption{FAST modell (Forrás: Kaufmann, 1982, \cite{Kaufmann:1982})}
    \label{fig:fast}
\end{figure}

\subsection{Hibafa analízis}
A Hibafa analízis (FTA\footnote{Fault Tree Analysis}) egy szisztematikus, deduktív és logaikai módszer a rendszerhiba (Top event) okainak feltárására, modellezésére, vizsgálatára.
Az FTA-t lehet az egyek legmegbízhatóbb eszköznek tekinteni a hiba eset logaikai kiértékelésében biztonsági és megbízhatósági szempontból. \cite{Ericson}

A hibafát H. Watson és Allison B. Mearns fejlesztette ki 1962-ben, akik együtt dolgoztak a US Bell Company laboratóriumában. Később a Boeing Company is elkezdte használni a módszert, hogy meghatározza a biztonsági faktorokat, melyek fegyverrendszerekre lehetnek hatással.
Az 1960-as években a kereskedelmi repülés, '70-es években a nukleáris energiaipar, '80-as években a vegyipar, '90-es években közlekedési ipar is elkezdte használni a módszert biztonságosság és megbízhatóság felmérésére (Ericson, 1999).

Az analízis az előfordulható hibamódok egy részhalmazára fokuszál, különös tekintettel azokra, amik katasztrofális hibát okozhatnak.
A kapuk (gate), események (event) és vágások (cut set) jelentik a legfőbb elemzési pontjait.
A logikai diagramm - 'ÉS' és 'VAGY' kapuk - adja meg a FTA eredményét.
A hibaesetek adják meg a kapuk bemeneteit és a vágások adják meg azoknak a hibaeseteknek halmazát, ami rendszerhibát okozhat.
Az FTA-t szokás az FMECA\footnote{Failure Mode Effect and Criticallity Analysis}-val, Markov analízissel és ETA\footnote{Event Tree analysis}-val együttesen használni, hogy elfedjék az FTA limitációit (Stapelberg, 2008; BE EN 60812, 2006).

\begin{figure}
    \footnotesize
    \centering
    \includegraphics[width=150mm, keepaspectratio]{figures/fta1.png}
    \caption{Egy hibafa példa pneumatikus fékrendszer meghibásodásához (Forrás: Long, 2017 \cite{Long2017BrakingSM})}
\end{figure}

\subsection{Megbízhatósági blokkdiagram analízis}
A Megbízhatósági blokkdiagram (RBD\footnote{Reliability Block Diagram}) egy vizuális elemzési módszer aminek segítségével könnyen reprezentálható a rendszer logikailag összekötött struktúrája.
Az RBD blokkjai ábrázolják a rendszer eredményes működését. Az elemzés különböző szinten jelentkezhet, mind kvalitatív, mind kvantitatív formában (BS EN 61078, 2006; BS EN 60300-3-1, 2004).

A diagramm felépíthető egyenesen a rendszer funkciónális modelljéből, ami szisztematikusan megjeleníti a funkcionális utakat.
Sok különböző rendszerkonfigurációt képes kifejezni, például, soros, párhuzamos, rendundáns, ,,standby'' stb., ahogy az az \ref{fig:rdb} ábrán is látszik.
Az RBD-t általában abban az esetben használják, amikor különböző változatait és komprumisszumait kell értékelni megbízhatósági és elérhetőségi szempontból.
\begin{figure}
    \footnotesize
    \centering
    \includegraphics[width=80mm, keepaspectratio]{figures/rbd1.png}
    \caption{Különböző RDB modellek (Forrás: BS EN 61078, 2006)}
    \label{fig:rdb}
\end{figure}

\subsection{Közös hibaforrás azonosítás}

\section{Bottom-up módszerek}
\subsection{FMEA analízis}
Az Hibamód és hatás analízis (FMEA\footnote{Failure Mode and Effect Analysis}) egy biztonságossági és megbízhatősági kiértékelő módszer, ami képes felmérni a rendszer összem komponensének összes potenciális hibamódját, ami kihathat az egész rendszer teljesítményére.
A módszer továbbá azonosítja azokat a módszereket is, amikkel elkerülhetők a hibamódok és hogyan lehetséges csökkenteni azok hatását.

A módszert eredetileg FMECA\footnote{Failure Mode Effect and Criticality Analysis} néven említették, amiben a ,C' betű a hibamód kritikusságát jellemezte.
Bár a két módszert gyakran szinonímaként használják, teljesen más a megközelítésük.
Általában az FMEA-t a hibamód hatásának súlyosságát minősíti, míg az FMECA a hibamód frekvenciáját is megvizsgálja a súlyosság mellett.
A súlyosság és a frekvencia kombinációját a rendszer kritikusságnak vagy kockázatának nevezik (MIL-STD-1629A, 1980).

A módszer során minimum a következő nyolc információt kell megállapítani:
\begin{enumerate}
    \item Hibamódok (Failure mode)
    \item A hibamód hatása a rendszerre
    \item A hibamód miatt bekövetkezett rendszerszintű hiba
    \item A veszélyek baleseti hatása
    \item A hibamód és/vagy veszély okozati tényezőit
    \item Hogyan lehet a hibamódot detektálni
    \item Javaslatok
    \item A felderített veszély kockázatát
\end{enumerate}

A módszert több szempontból is el lehet végezni.
Ez lehetnek Funkcionális megközelítés, Struktúrális megközelítés és a ,,hibrid'' megoldás.
Az első megközelítésben a funkciók céljának lehetséges hibás működését veszi figyelembe.
Ez a módszer alkalmazható szoftverek esetében is.

A második megközelítés leginkább hardver elemeken végzett lehetséges hibamódokra öszpontosít.
A ,,hibrid'' megközelítés először a funkcionális analízissel kezdődik, aminek fókusza átvált a hardverre (Ericson, 2005).

Lehetséges példák a funkcionális hibamódokra:
\begin{itemize}
    \item A funkció nem működik
    \item A funkció nem megfelelően működik
    \item A funkció idő előtt hajtódik végre
    \item A funkció hibás vagy félrevezető információt szolgáltat
    \item A funkció nem hibásodik meg biztonságosan
\end{itemize}
Lehetséges hardver hibamód kategóriák:
\begin{itemize}
    \item Teljes meghibásodás
    \item Részleges meghibásodás (például: tolerancián kívül)
    \item Időszakos meghibásodás
\end{itemize}
Lehetséges hardver hibamódok:
\begin{itemize}
    \item Szakadás
    \item Rövidzárlat
    \item Tolerancián kívül
    \item Szivárgás
    \item Meleg felület
    \item Elhajlás
    \item Túl/alulméretezett
    \item Megrepedt
    \item Rideg
    \item Elmozdult
    \item Korrodált
    \item stb.
\end{itemize}
Lehetséges hibamódok szoftvereknél:
\begin{itemize}
    \item A szoftver funkció meghibásodik
    \item A funkció hibás eredményt szolgáltat
    \item A funkció idő előtt meghívódik
    \item Elküldetlen üzenetek
    \item Túl korán/későn küldött üzenet
    \item Hibás üzenet
    \item Megáll vagy összeomlik a szoftver
    \item Belső kapacitásoknál többet igényel a szoftver
    \item Szoftver ,,startup'' hiba
    \item Lassú válaszidő
\end{itemize}

\subsection{HAZOP analízis}

\subsection{LOPA}
Layer of Protection Analysis (LOPA) \cite{LOPA1}