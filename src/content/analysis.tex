%----------------------------------------
\chapter{\analysis}\label{chap:analysis}
%----------------------------------------
A dolgozat során elkészített rendszer terv csak akadémiai célokat szolgál, ezért nem követi teljes mértékben a valóságot.
Teljes mértékben kielégítő biztonsági vizsgálatot ezért nem lehet végezni a rendszeren.
A következő részekben inkább szeretném bemutatni egyes elemzési módszerek lényegét, alkalmazhatóságát mintsem egy valós projekt szerinti precíz megvalósítsát.

Mivel a biztonságosság kiemelten fontos témakör az iparágban, ezért az elemzésben a valósághoz hasonló értékek szerepelnek, de ezen értékek kitaláltak és reprezentatív jellegűek.
Az analízis során feltárt kockázatok besorolását az EN 50126\cite{EN50126-1} első részében leírt kockázat mátrix kalibrációs táblázatai szerint határoztam meg.

A frekvenciák besorolásához a \ref{tab:freqency}. táblázatot készítettem el.
Ennek segítségével besorolhatók egyes kvantitatív értékek az elfogadási mátrixban.

Továbbá a \ref{tab:serverity} és a \ref{tab:risk} segédtáblázatok alapján már meghatározható az a táblázat, amely a balesetek súlyossága és a hiba előfordulásának gyakorisága alapján meghatározza a kockázat elfogadásának lehetőséget.
Ezt a táblázatot reprezentálja a \ref{tab:risk_mat} táblázat.

Az elemzések további szakaszaiban ezen táblázatok felhasználása szerint lesznek meghatározva az előforduló hibák.
\begin{table}
    \footnotesize
    \centering
    \begin{tabular}{ |c|p{10em}|p{8em}|p{10em}| }
        \hline
        Frekvencia szint & Leírás & Frekvencia példa egy 24 órában működő eszköznél & Frekvencia \\
        \hline
        Gyakori & Az esemény gyakran bekövetkezik & Több mint egyszer 6 hónapos időszak alatt & $f \leq 1\times10^{-3}$ \\
        \hline
        Valószínű & Az esemény várhatóan többször bekövetkezik & Hozzávetólegesen egyszer 6 hét és egy év között & $ 1\times10^{-3} < f \leq 1\times10^{-4} $ \\
        \hline
        Alkalmi&Az esemény több alkalommal bekövetkezhet & Hozzávetőlegesen egyszer egy év és 10 év között & $1\times10^{-4} < f \leq 1\times10^{-5}$ \\
        \hline
        Ritka&Valószínű, hogy bekövetkezik valamikor a rendszer élettartama során & Hozzávetőlegesen egyszer 10 év és 1 000 év között & $1\times10^{-5} < f \leq 1\times10^{-7}$ \\
        \hline
        Valószínűtlen&Velószínűtlen a bekövetkezés, de megtörténhet & Hozzávetőlegesen egyszer 1 000 év és 100 000 év között & $1\times10^{-7} < f \leq 1\times10^{-9}$ \\
        \hline
        Erősen valószínűtlen&Az esemény vélhetően egyszer sem fog bekövetkezni & Hozzávetőlegesen egyszer 100 000 éveben vagy kevesebbszer & $1\times10^{-9} < f$ \\
        \hline
    \end{tabular}
    \caption{A veszély besorolási mátrix frekvencia tartomány kalibrációja. Saját adatok, a \cite{EN50126-1} alapján.}
    \label{tab:freqency}
\end{table}

\begin{table}
    \footnotesize
    \centering
    \begin{tabular}{ |c|p{10em}|p{10em}| }
        \hline
        Súlyossági kategória & Következmények az emberekre / környezetre                                                                                                     & Következmények a szolgáltatásokra \\ \hline
        Katasztrofális       & Emberek tömegét befolyásolja és többek halálához vezethez és/vagy  súlyos környezeti károsodás      & -                                 \\ \hline
        Kritikus             & Emberek kis számát befolyásolja és legalább egy halálesetet okoz és/vagy  nagy környezeti károsodás & Egy fontos rendszer elvesztése    \\ \hline
        Marginális           & Nincs lehetőség halálesetre, cask súlyos vagy könnyű sérülések és/vagy kis környezeti károsodás     & Súlyos rendszer sérülés           \\ \hline
        Jelentéktelen        & Lehetséges könnyű sérülés                                                                                                                     & Jelentéktelen rendszer sérülés    \\ \hline
    \end{tabular}
    \caption{Súlyossági kategóriák, az EN 50126-1 alapján.}
    \label{tab:serverity}
\end{table}

\begin{table}
    \footnotesize
    \centering
    \begin{tabular}{ |c|p{20em}| }
        \hline
        Kockázat elfogadási kategória & Végrehajtandó intézkedések\\ \hline
        Tolerálhatatlan & A kockázatot meg kell szüntetni \\ \hline
        Nem kívánatos & A kockázatot csak abban az esetben lehet elfogadni, ha annak csökkentése nem lehetséges \\ \hline
        Tolerálható & A kockázat tolerálható és elfogadható megfelelő ellenőrzéssel (például: karbantartási útmutatók vagy szabályok) \\ \hline
        Elhanyagolható & A kockázat minden további nélkül elfogadható \\ \hline
    \end{tabular}
    \caption{Kockázat elfogadási kategóriák, az EN 50126-1 alapján.}
    \label{tab:risk}
\end{table}

\begin{table}
    \centering
    \begin{tabular}{|l|l|l|l|l|} 
        \hline
        Előfordulási frekvencia & \multicolumn{4}{l|}{Kockázat elfogadási kategória}                    \\ 
        \hline
        Gyakori                 & Nem kívánatos  & Tolerálhatatlan & Tolerálhatatlan & Tolerálhatatlan  \\ 
        \hline
        Valószínű               & Tolerálható    & Nem kívánatos   & Tolerálhatatlan & Tolerálhatatlan  \\ 
        \hline
        Alkalmi                 & Tolerálható    & Nem kívánatos   & Nem kívánatos   & Tolerálhatatlan  \\ 
        \hline
        Ritka                   & Elhanyagolható & Tolerálható     & Nem kívánatos   & Nem kívánatos    \\ 
        \hline
        Valószínűtlenű          & Elhanyagolható & Elhanyagolható  & Tolerálható     & Nem kívánatos    \\ 
        \hline
        Erősen valószínűtlen    & Elhanyagolható & Elhanyagolható  & Elhanyagolható  & Tolerálható      \\ 
        \hline
        \multicolumn{1}{l|}{}   & Jelentéktelen  & Merginális      & Kritikus        & Katasztrofális   \\ 
        \cline{2-5}
        \multicolumn{1}{l|}{}   & \multicolumn{4}{l|}{A baleset súlyossága}                             \\
        \cline{2-5}
        \end{tabular}
    \caption{Kockázat elfogadási mátrix.}
    \label{tab:risk_mat}
\end{table}

\section{HAZOP analízis}
A vizsgálatot HAZOP tanulmány készítésével kezdtem.
Ez alkalmas a rendszer platform modellje alapján előre meghatározott módszertan szerint szisztematikusan eltárni az adott alrendszerek tervezettől való eltérő viselkedéseinek feltárására.

A vizsgálat segéd szavak segítségével vezeti rá az elemző (csapat) gondolkodását a lehetséges eltérésekre.
A módszertan által nyújtott lehetséges szavak csak egy részét felhasználva, a következő iránymutató szavak lettek felhasználva a dolgozatban: (1) NEM (NO), (2) KEVESEBB (LESS), (3) TÖBB (MORE).

Ezután két alrendszert vizsgáltam a módszertan szerint, ezek pedig a hidraulikus fékolló (Hydraulic Caliper) és a vezérlő elektronika (Control Electronics).

A HAZOP tanulmány eredményessége nagyban függ az azt végrehajtó csapat kreativitásán, ezért nem feltételezhető az adott egység minden egyes hibaforrásának észlelése/feltárása.

Az általam elvégzett vizsgálat során a \ref{tab:hazop_vizsg}. táblázatban látható potenciális eltérésekre jutottam.


\begin{table}
    \centering
    \begin{tabular}{ |p{18mm}|p{15mm}|p{20mm}|p{20mm}|p{25mm}|p{25mm}| }
        \hline
        Item & Guide word & Deviation & Cause & Consequence & Existing controls \\
        \hline
        Hydraulic Caliper & NO & No brake force & Caliper failure & Longer braking distance / possible crash with traffic & Recommended service interval / routine checks \\
        & LESS & Reduced brake force & Worn out brakepads & Longer braking distance / possible crash with traffic & Recommended service interval / routine checks \\
        & LESS & Reduced brake force & Control loop failure & Longer braking distance / possible injury to multiple people & - \\
        & MORE & More brake force & Failure of pressure generation & Possibility of train stuck on track / minor injuries & - \\
        \hline
        Control Electronics & NO & No input voltage & Train battery line failure & Train stuck on track & - \\
        & MORE & Higher input voltage than specified & Train battery line failure & Control electronics power supply failure & Over voltage protection \\
        & MORE & Higher brake demand than requested & Control output stuck at low & Possibility of train stuck on track / minor injuries & Under voltage protection \\
        & LESS & Lower input voltage than specified & Train battery line failure & Possibility of train stuck on track / minor injuries & - \\
        & LESS & Lower brake demand than requested & Control output stuck at high & Longer braking distance / possible injury to multiple people & Error signaling \\
        & LESS & Lower brake demand than requested & Wheelslide protection lowers the brake demand for a long time & Longer braking distance / possible injury to multiple people & WSP watchdog \\
        \hline
    \end{tabular}
    \caption{HAZOP vizsgálat a hidraulikus fékolló és vezérlő elektronika komponensekre.}
    \label{tab:hazop_vizsg}
\end{table}

\section{LOPA analízis}

\section{FTA}