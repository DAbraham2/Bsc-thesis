\section{FIT Allokáció} \label{sec:fit_allocation}
Az ipari rendszerek egyre összetettebbekké váltak az évek során. 
Emellett mostanság egyre több ilyen rendszer tartalmaz elektronikát és szoftvert, 
tehát a funkcionális biztonságnak folyamatosan nő a fontossága. (Schabe, 2018 \cite{Schabe})

A safety integrity level (SIL) egy diszkrét érték ami meghatározza a használandó módszereket, technikákat a véletlenszerű és szisztematikus hibák elkerülése érdekében.
A SIL-ek koncepciója már több szabvány rendszerben ki lett fejlesztve.
Ezek között legismertebb szabványok az IEC 61508, DEF-STAN-0056, EN 50126, EN 50128, EN 50129 és még sok más.

A SIL-nek két fő aspektusa van:
\begin{enumerate}
    \item Egy cél hibaráta, amit a rendszernek nem szabad meghaladni, hogy tudja kezelni a véletlen hibákat
    \item Módszerek és technikák halmaza, amit a szisztematikus hibákat kezeli 
\end{enumerate}

Itt fontos megjegyezni, hogy szoftverben csak és kizárólag szisztematikus hibákat vesznek figyelembe és nincs megadva cél hibaráta. Ez abból adódik, hogy a szoftverben nincs véletlenszerű hiba.

\subsection{Különböző SIL-ek}
A \ref{tab:SILs} táblázat példát ad a SIL-ek és a tűrhető hibaráta kapcsolatára, ahogy három szabvány, az IEC 61508, az EN 50129 és a DEF-STAN-0056 definiálja.

A tűrhető hibaráta (THR) egy eszköz veszélyes hibáinak maximális rátája, amit a szabvány definiál bizonyos safety integrity szinthez.
Látni kell azt, hogy bár az IEC és EN szabványoknál azonosak a THR értékek, a DEF-STAN-0056-ban eltér.
Ezért a szabványok közti átjárás nem mindig triviális.
Attól még, hogy a THR értékek hasonlóak az IEC és EN szabványok között, az rendszer szintű hibaelkerülő módszerek különböznek, ezért ezek a SIL-ek sem ugyan azok.
\begin{table}[ht]
	\footnotesize
	\centering
	\begin{tabular}{ l c c }
		\toprule
		SIL & IEC 61508/EN 50129 & DEF-STAN-0056 \\
		\midrule
		1 & \(10^{-6}/h \leq\) THR \(< 10^{-5}/h\) & Frequent \(\approx 10^{-2}/h\)\\
		2 & \(10^{-7}/h \leq\) THR \(< 10^{-6}/h\)  & Probable \(\approx 10^{-4}/h\)\\
		3 & \(10^{-8}/h \leq\) THR \(< 10^{-7}/h\)  & Occasional \(\approx 10^{-6}/h\)\\
		4 & \(10^{-9}/h \leq\) THR \(< 10^{-8}/h\)  & Remote \(\approx 10^{-8}/h\) \\
		\bottomrule
	\end{tabular}
	\caption{SIL értékek több szabvány és THR szerint}
	\label{tab:SILs}
\end{table}

\subsection{Safety integrity szintek kombinációja}
Ebben a részben a különböző szabványok SIL szintjeinek kombinációját mutatom be.

\subsubsection{DEF-STAN-0056}
