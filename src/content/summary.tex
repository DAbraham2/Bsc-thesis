\chapter{Összefoglalás}
A szakdolgozat során felkutattam számtalan a biztonsági analízishez használt módszert.
Ezen módszerek nagyrésze már szerepel a vasútipari szabványok valamelyikében, de más iparából is származott hasznos technológia.

A módszerek között megkülönböztettem lehetséges Top-down és Bottom-up irányú eszközöket is, leírva azok előnyét a fejlesztés lépéseinek támogarása során.

A dolgozat során bemutattam a modell-alapú rendszerfejlesztés pár lehetőségét a vasútiparra tekintve.
Elkészült a fékrendszer funkcionális modellje/architektúrája.
A fejlesztés során bemutattam a funkcionális dekompozíció lépéseit és ezáltal eljutottam egy kinevezett funkció (üzemi fék) alacsonyabb szintjeihez.

A funkcionális dekompozíció Top-down metodikája mellett elkészült a fékrendszer egy platform modellje.
Ebben bemutattam a Bottom-up módszertan felépítését, majd a kis komponensekből bemutattam a felépített rendszer modelljét.
Ezen a platform architektúrán már megjelentek a fizikai építőelemek és azok összeköttetéseik.

A modellezés végeztével az általam választott módszerekkel biztonsági analízist hajtottam végre az elkészült terveken.

Ebbeb bemutattam a HAZOP, a LOPA és az FTA módszereket.
A HAZOP egy magasszintű szisztematikus problémafeltáró eljárás, amely iránymutató szavak segítségével vezetett rá a lehetséges hibákra.
A LOPA átmenet a HAZOP és az FTA között. Segítségével egy egyszerű hibát rangsoroltam.
Az FTA összetett hibák elemzésére alkalmas eszköz aminek segítségével egy komplex alegység biztonsági vizsgálatát végeztem el.

\section{Javaslatok további munkára}
A jövőben szeretném részletesebben kidolgozni a rendszertervet.
Ebbe beletartozik a funkciók részletesebb dekompozíciója és ezen funkciók viselkedésének/kapcsolatainak modellezése.
Érdekes lehet még valós adatok/követelmények alapján készíteni a modelleket különböző tervezési kényszer mellett.

Továbbá szeretném tovább részletezni a biztonsági analízist is.
Ennél a feladatkörnél is érdekes lehet valós adatok/tények felkutatása és azok alapján végezni a vizsgálatot.