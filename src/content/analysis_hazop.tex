\section{HAZOP analízis}
A vizsgálatot HAZOP tanulmány készítésével kezdtem.
Ez alkalmas a rendszer platform modellje alapján előre meghatározott módszertan szerint szisztematikusan eltárni az adott alrendszerek tervezettől való eltérő viselkedéseinek feltárására.

A vizsgálat segéd szavak segítségével vezeti rá az elemző (csapat) gondolkodását a lehetséges eltérésekre.
A módszertan által nyújtott lehetséges szavak csak egy részét felhasználva, a következő iránymutató szavak lettek felhasználva a dolgozatban: (1) NEM (NO), (2) KEVESEBB (LESS), (3) TÖBB (MORE).

Ezután két alrendszert vizsgáltam a módszertan szerint, ezek pedig a hidraulikus fékolló (Hydraulic Caliper) és a vezérlő elektronika (Control Electronics).

A HAZOP tanulmány eredményessége nagyban függ az azt végrehajtó csapat kreativitásán, ezért nem feltételezhető az adott egység minden egyes hibaforrásának észlelése/feltárása.

Az általam elvégzett vizsgálat során a \ref{tab:hazop_vizsg}. táblázatban látható potenciális eltérésekre jutottam.


\begin{table}
    \centering
    \begin{tabular}{ |p{18mm}|p{15mm}|p{20mm}|p{20mm}|p{25mm}|p{25mm}| }
        \hline
        Item & Guide word & Deviation & Cause & Consequence & Existing controls \\
        \hline
        Hydraulic Caliper & NO & No brake force & Caliper failure & Longer braking distance / possible crash with traffic & Recommended service interval / routine checks \\
        & LESS & Reduced brake force & Worn out brakepads & Longer braking distance / possible crash with traffic & Recommended service interval / routine checks \\
        & LESS & Reduced brake force & Control loop failure & Longer braking distance / possible injury to multiple people & - \\
        & MORE & More brake force & Failure of pressure generation & Possibility of train stuck on track / minor injuries & - \\
        \hline
        Control Electronics & NO & No input voltage & Train battery line failure & Train stuck on track & - \\
        & MORE & Higher input voltage than specified & Train battery line failure & Control electronics power supply failure & Over voltage protection \\
        & MORE & Higher brake demand than requested & Control output stuck at low & Possibility of train stuck on track / minor injuries & Under voltage protection \\
        & LESS & Lower input voltage than specified & Train battery line failure & Possibility of train stuck on track / minor injuries & - \\
        & LESS & Lower brake demand than requested & Control output stuck at high & Longer braking distance / possible injury to multiple people & Error signaling \\
        & LESS & Lower brake demand than requested & Wheelslide protection lowers the brake demand for a long time & Longer braking distance / possible injury to multiple people & WSP watchdog \\
        \hline
    \end{tabular}
    \caption{HAZOP vizsgálat a hidraulikus fékolló és vezérlő elektronika komponensekre.}
    \label{tab:hazop_vizsg}
\end{table}